\documentclass[runningheads,a4paper]{llncs}

\usepackage{amssymb}
\setcounter{tocdepth}{3}
\usepackage{graphicx}
\usepackage{hyperref}
\usepackage{fixltx2e}

\usepackage{url}
\urldef{\mailsa}\path|{i.c.t.m.speek, a.c.stolwijk}@student.tudelft.nl|
\newcommand{\keywords}[1]{\par\addvspace\baselineskip%
\noindent\keywordname\enspace\ignorespaces#1}

\begin{document}

\mainmatter% start of an individual contribution

% first the title is needed
\title{Paper Plan\\
FF-Replan: \& RFF\@: Exploiting classical AI planning for uncertain and probabilistic domains}

% a short form should be given in case it is too long for the running head
\titlerunning{Presentation Plan}

\author{I.C.T.M Speek, A.C. Stolwijk}

%
\authorrunning{Paper Plan Seminar Algorithms FF-replan \& RFF}
% (feature abused for this document to repeat the title also on left hand pages)

% the affiliations are given next; don't give your e-mail address
% unless you accept that it will be published
\institute{Seminar Algorithms, Embedded Software,
Msc Embedded Systems,\\
Delft University of Technology\\
\mailsa\\
}

%\toctitle{Lecture Notes in Computer Science}
%\tocauthor{Authors' Instructions}

\maketitle

%--------------------------------------------------------------------

\section{Introduction}

\begin{itemize}
	\item Introduce planning problem
	\item Introduce planning for uncertainty
	\item Introduce the problem definition and motivation
\end{itemize}

We will focus on 2 of the current state of the art probabilistic planning
algorithms that apply determinization in probabilistic domains and criticize
the approaches by formulating a real-world domain as opposed to the IPPC
domains.  By comparing both approaches and analyzing other state of the art
work we will propose some adjustments which we will theoretically test in this
domain. If it is realistic, we will also try to create an empirical evaluation.

%---------------------------------------------------------------------

\section{Related Work}

\begin{itemize}
	\item Introduce the previously read papers
	\item Introduce the current state of the art papers that use deterministic approaches for probablistic domains deadling with uncertainty
\end{itemize}

\subsubsection{Probabilistic Planning via Determinization in Hindsight (2008)~\cite{yoon2008probabilistic}}

Hindsight optimization is an online technique that evaluates one-step-reachable
states by sampling future outcomes to generate multiple non-stationary planning
problems which are deterministic and can be used using search. It re-interprets
FF-Replan's~\ref{FFReplan} strategy randomly generating a set of non-stationary
determinized problems and combining their solutions.

\subsubsection{Probabilistic Planning in the Graphplan Framework (1999)~\cite{Blum99probabilisticplanning}}

Graphplan is a successful planning algorithm for classical STRIPS domains. This
paper explores the extend it can be used in \emph{probabilistic} domains. The
paper discusses two variations of Graphplan: PGraphplan and TGraphplan.
PGraphplan produces an optimal plan, while TGraphplan produces a sub-optimal
plan but it has increased speed. By comparing the speed and quality of the
two planners the authors are able to estimate how far they are from the ideal.

\subsubsection{Compiling Conformant Probabilistic Planning Problems into Classical Planning (2013)~\cite{taig2013conformant}}

\subsubsection{How Much Does a Household Robot Need To Know In Order To Tidy Up? (2013)~\cite{nebel2013much}}

For household robot planning, it appears easy, but actually involves
uncertainty. For tidying things up objects might not be there, or sensing
operations might tell things wrong. This paper looks into conditions for
classical planning in a replanning loop in order to solve nondeterministic
partially observable open domains This paper looks into conditions for
classical planning in a replanning loop in order to solve nondeterministic
partially observable open domains.

\subsubsection{Progressive heuristic search for probabilistic planning based on interaction estimates (2013)~\ref{martin2013progressive}}

In this paper a probabilistic plan graph heuristic is described which computes
information about the interaction between actions and between propositions.
This information is used to find better relaxed plans to compute the
probability of success. This information guides a forward state space search
for high probability, non-branching seed plans. These plans are then used in
a planning and scheduling system that handles unexpected outcomes by runtime
replanning.

\subsubsection{Translation based approaches to probabilistic planning (2013)~\cite{taig2013translation}}

%---------------------------------------------------------------------

\section{Problem formulation}

\begin{itemize}
	\item Summarize shortcomings of the previous solutions
	\item Describe the planning problem
	\item Argument the choice for 2 current papers to analyze
	\item Define how we would like to test the planning algorithms in a domain by our choice
\end{itemize}

%---------------------------------------------------------------------

\section{Domain description}

\begin{itemize}
	\item Describe domains from IPPC
	\item Explain shortcomings
	\item Evaluate a valuable real-world domain in which we can test 2 papers
\end{itemize}

%---------------------------------------------------------------------

\section{Theoretical approach}

\begin{itemize}
	\item Describe how we are using the created domain
	\item show pseudo code for the 2 papers
	\item Introduce the improvements (or combination) of the approaches of the papers fitted for the domain
\end{itemize}

%---------------------------------------------------------------------

\section{Empirical evaluation}

\begin{itemize}
	\item Describe the experiment setup
	\item Show results for the experiment
	\item Critically analyze the results
	\item Compare with results from the original papers
\end{itemize}

%---------------------------------------------------------------------

\section{Conclusion}

\begin{itemize}
	\item Conclude on the success of the 2 approaches
	\item Conclude the work done in the paper
	\item Hint to some future work developments
\end{itemize}

\section{Some papers}

\begin{itemize}
	\item \cite{Hoffmann01theff}
	\item \cite{FFReplan}
	\item \cite{teichteil2010incremental}
	\item \cite{yoon2008probabilistic}
	\item \cite{teichteil2012fast}
	\item \cite{bonet2011planning}
	\item \cite{taig2013translation}
	\item \cite{martin2013progressive}
	\item \cite{taig2013conformant}
	\item \cite{2013workshop}
\end{itemize}


\bibliographystyle{plain}
\bibliography{paper-plan}

\end{document}
