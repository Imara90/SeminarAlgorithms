\documentclass[runningheads,a4paper]{llncs}

\usepackage{amssymb}
\setcounter{tocdepth}{3}
\usepackage{graphicx}
\usepackage{hyperref}
\usepackage{fixltx2e}

\usepackage{url}
\urldef{\mailsa}\path|{i.c.t.m.speek, a.c.stolwijk}@student.tudelft.nl|
\newcommand{\keywords}[1]{\par\addvspace\baselineskip%
\noindent\keywordname\enspace\ignorespaces#1}

\begin{document}

\mainmatter% start of an individual contribution

% first the title is needed
\title{Paper Plan\\
FF-Replan: \& RFF\@: Exploiting classical AI planning for uncertain and probabilistic domains}

% a short form should be given in case it is too long for the running head
\titlerunning{Presentation Plan}

\author{I.C.T.M Speek, A.C. Stolwijk}

%
\authorrunning{Paper Plan Seminar Algorithms FF-replan \& RFF}
% (feature abused for this document to repeat the title also on left hand pages)

% the affiliations are given next; don't give your e-mail address
% unless you accept that it will be published
\institute{Seminar Algorithms, Embedded Software,
Msc Embedded Systems,\\
Delft University of Technology\\
\mailsa\\
}

%\toctitle{Lecture Notes in Computer Science}
%\tocauthor{Authors' Instructions}

\maketitle

\section{Motivation}

There is some domain in which requires probabilistic planning. As FF-Replan and
RFF are about using deterministic planners in a probabilistic domain, we like
to see what's new and if we can find some holes that are still open.

\section{Related Work}

\subsection{Probabilistic Planning in the Graphplan Framework (1999)~\cite{Blum99probabilisticplanning}}

Graphplan is a successful planning algorithm for classical STRIPS domains. This
paper explores the extend it can be used in \emph{probabilistic} domains. The
paper discusses two variations of Graphplan: PGraphplan and TGraphplan.
PGraphplan produces an optimal plan, while TGraphplan produces a sub-optimal
plan but it has increased speed. By comparing the speed and quality of the
two planners the authors are able to estimate how far they are from the ideal.

\subsection{Compiling Conformant Probabilistic Planning Problems into Classical Planning (2013)~\cite{taig2013conformant}}

\section{Conclusion from Related Work and problem statement}

From the related literature we identify some problems.

\section{Solution for the problem}

We try to find some solution or direction how to solve the problem

\section{Conclusion}

Did it work.

\section{More Literature}

\begin{itemize}
	\item \cite{Hoffmann01theff}
	\item \cite{FFReplan}
	\item \cite{teichteil2010incremental}
	\item \cite{yoon2008probabilistic}
	\item \cite{teichteil2012fast}
	\item \cite{bonet2011planning}
	\item \cite{taig2013translation}
	\item \cite{martin2013progressive}
	\item \cite{taig2013conformant}
	\item \cite{2013workshop}
\end{itemize}


\bibliographystyle{plain}
\bibliography{paper-plan}

\end{document}
