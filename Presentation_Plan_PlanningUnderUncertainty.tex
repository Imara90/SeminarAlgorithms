%%%%%%%%%%%%%%%%%%%%%%% file ff-replan.tex %%%%%%%%%%%%%%%%%%%%%%%%
%
% 	Presentation Plan Arian Stolwijk & Imara Speek
%
%%%%%%%%%%%%%%%%%%%%%%%%%%%%%%%%%%%%%%%%%%%%%%%%%%%%%%%%%%%%%%%%%%%

\documentclass[runningheads,a4paper]{llncs}

\usepackage{amssymb}
\setcounter{tocdepth}{3}
\usepackage{graphicx}

\usepackage{url}
\urldef{\mailsa}\path|{i.c.t.m.speek, a.c.stolwijk}@student.tudelft.nl|
\newcommand{\keywords}[1]{\par\addvspace\baselineskip%
\noindent\keywordname\enspace\ignorespaces#1}

\begin{document}

\mainmatter% start of an individual contribution

% first the title is needed
\title{Presentation Plan\\
FF-Replan: \& RFF\@: Exploiting classical AI planning for uncertain and probabilistic domains}

% a short form should be given in case it is too long for the running head
\titlerunning{Lecture Notes in Computer Science: Authors' Instructions}

\author{I.C.T.M Speek, A.C. Stolwijk}

%
\authorrunning{Presentation Plan Seminar Algorithms FF-replan \& RFF}
% (feature abused for this document to repeat the title also on left hand pages)

% the affiliations are given next; don't give your e-mail address
% unless you accept that it will be published
\institute{Seminar Algorithms, Embedded Software,
Msc Embedded Systems,\\
Delft University of Technology\\
\mailsa\\
}

\toctitle{Lecture Notes in Computer Science}
\tocauthor{Authors' Instructions}
\maketitle

\section{Summary of given papers}

At the first international probabilistic planning competition IPPC-04, most
entries were Markov Decision Processes. The winner, FF-Replan, however was
based on deterministic planning techniques.

By relying on search, FF-Replan could function with enormous speed and efficiency.

Until the paper there have been two IPPC conferences.

\section{Questions that arose}

\section{Criticism on the contribution of the paper}

\section{Relevant literature on the subject}

\section{Presentation setup}

\end{document}
