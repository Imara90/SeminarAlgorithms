\documentclass[runningheads,a4paper]{llncs}

\usepackage{amssymb}
\setcounter{tocdepth}{3}
\usepackage{graphicx}
\usepackage{hyperref}
\usepackage{fixltx2e}

\usepackage{url}
\urldef{\mailsa}\path|{i.c.t.m.speek, a.c.stolwijk}@student.tudelft.nl|
\newcommand{\keywords}[1]{\par\addvspace\baselineskip%
\noindent\keywordname\enspace\ignorespaces#1}

\begin{document}

\mainmatter% start of an individual contribution

% first the title is needed
\title{Presentation Plan\\
FF-Replan: \& RFF\@: Exploiting classical AI planning for uncertain and probabilistic domains}

% a short form should be given in case it is too long for the running head
\titlerunning{Presentation Plan}

\author{I.C.T.M Speek, A.C. Stolwijk}

%
\authorrunning{Presentation Plan Seminar Algorithms FF-replan \& RFF}
% (feature abused for this document to repeat the title also on left hand pages)

% the affiliations are given next; don't give your e-mail address
% unless you accept that it will be published
\institute{Seminar Algorithms, Embedded Software,
Msc Embedded Systems,\\
Delft University of Technology\\
\mailsa\\
}

%\toctitle{Lecture Notes in Computer Science}
%\tocauthor{Authors' Instructions}

\maketitle

\section{Introduction to a planner (A)}

What is a planner? [Will need some more detailing].\\

Examples of planning problems:

\begin{itemize}
	\item Blocks World
	\item Motion Planning
\end{itemize}

Planning Problem:

\begin{itemize}
	\item Initial State
	\item Goal State
	\item Set of actions
\end{itemize}

Planning Plan:

\begin{itemize}
	\item Set of actions
	\item Executable in the initial state
	\item Resulting in a state that satisfies the goal state
\end{itemize}

\section{What is uncertainty (I)}

\emph{Uncertainty applies to predictions of future events, to physical measurements that are already made, or to the unknown. Uncertainty arises in partially observable and/or stochastic environments, as well as due to ignorance and/or indolence} [1]. \\

Uncertain (adjective). \\
1. not able to be relied on; not known or definite. [2]

In the rest of this topic discuss the different domains and applications.

\subsection{Real world examples}

Growing plants is an extremely uncertain domain. When everything goes well, it should be fine to water it once a week. However my room environment has proven to be an uncertain domain where wind, temperature and sunlight severely impacts my abality to plan for nurturing my plants.

\subsection{Embedded Systems in uncertain domains}

Cruise control (stabilizing systems), quad rotor, AI, robots

\subsection{Benchmarking for uncertain domains}

Traffic control benchmark for the 2014 IPCC probabilistic continuous domain.
Tidybot: a housecleaning robot [3]. \\

\noindent [1] [Wikipedia http://en.wikipedia.org/wiki/Uncertainty]
[2] [Google definition]
[3] [http://www.plg.inf.uc3m.es/ipc2011-deterministic/DomainsSequential.html\#Barman]

\section{Historic Overview}
A swift introduction to the evolution of planning algorithms. For every problem state the:
\begin{itemize}
	\item The problem definition
	\item The motivation for the solution
	\item The strengths
	\item The weaknesses
\end{itemize}

\subsection{Classical Planners}

\subsection{Planning graphs}

\subsection{Planning as a satisfiability}

\subsection{Heuristic search Planning (HSP)}

\subsection{FF Planner (I)}
Use this: \cite{Hoffmann01theff}

\subsection{Markov Decision Processes}

State what's \cite{monahan1982state} is saying about the state of the art MDPs
in the 1980s.

\section{Planning under uncertainty}

\subsection{FF-replan \cite{FFReplan}}
%Define the problem considering probabilistic and deterministic planning

\subsubsection{Problem definition} 
Probabilistic planning is naturally formulated using Markov Decision Processes. These probabilistic planners focus on the entire state space. Deterministic planning has developed heuristic funtions enabling exploration of restricted portions of the state space enroute to the goal state. This enables the planner to account for unexpected states rather than ending up in a dead-end. MDP's had been the standard in planning algorithms [FOR HOW LONG]? However by relying on search and current available processing power, deterministic planners function with enormous speed and efficiency. This has been a motivation for combining deterministic and probabilistic planners in the planning communities. 

\subsubsection{Motivation for the solution}
FF-Replan is an action selection algorithm for online planning in probabilistic domains. An important notice is that the authors of the paper merely hoped to promote cross-fertilization of deterministic planning techniques in probabilistic planning. The basics of the FF-replan algorithm are simple and are valuable mostly because of the speed and efficiency introduced by relying on search.  It determinizes the input domain, removing all probabilistic information from the problem and synthesizes a plan. During the execution of this plan, should an unexpected state occur, the planner replans in the same determinization. 

\subsubsection{Details of the solution}
FF-Replan used two approaches to determinize the input domain. The first is single-outcome determinization which selects one outcome for each probabilistic construct. The heuristics determine the performance of the planner and should be chosen with care. The second approach uses an all-outcomes determinization and considers every probabilistic outcome as a distinct deterministic action. FF-Replan than generates one action for each effect (recursively). This ensures that an action in a state is chosen that is the first action of a sequential plan with non-zero probability of reaching the goal. 

The planner maintains a partial state-action mapping using a hash-table. It determinizes the states and synthesizes a plan using FF and then places the state-action sequence in the table. 

It doesn't deal with quantified goals but it picks an arbitrary grounded form of any existential goal. 

\subsubsection{Succes of the solution}
The planner performs strong on Logistic style domains and dead-end free domains. It performs very fast by relying on search. 

\subsubsection{Shortcomings of the solution}
\begin{enumerate}
	\item When no info about the probabilistic effects is provided, all effects are treated as equal. The FF-replan then doesn't attempt to avoid actions that can potentially end up in a dead end or away from the goal.
	\item  The planner only works well if the outcomes are likely. It might behave poorly in a very fluctuating behavioural environment. 
	\item The partial state-action policy does not guarantee quality. It may actually loop forever.  
	\item Picking arbitrary grounded forms of the goal may end up being unreachable or a lot more expensive to reach.
	\item The single outcome heuristic does not select the dead end outcomes and thus is completely unaware of them.  
\end{enumerate}

\subsubsection{Points of improvement}
\begin{enumerate}
	\item
	\item 
	\item Adaptively repair the partial policy 
	\item Sampling of the grounded goals and testing with a relaxed reachability analysis
	\item This is solved by using the all-outcomes approach
	\item It does not look ahead, but only adjusts when something unexpected has occured

	\item Investigate incremental planning approaches that leverage the most deterministic plan
\end{enumerate}

\subsubsection{Things to look into}
\begin{itemize}
	\item How is the exploding block domain currently tackled as during 2004 and 2006 no one really was able to solve it?
	\item Is the planning time still a problem as much as it proved to be then with the bigger domains?
	\item Using hindsight optimization might be more realistic as it is less optimistic. It will eventuall increase the process time, but increase the robustness of the system. 
	\item Using a look ahead policy enables the planner to avoid dead ends, it however does not avoid cycles.
\end{itemize}

\subsection{RFF (A)}

\subsection{Compare FF-Replan with RFF (I\&A)}

\subsection{How did FF-Replan and RFF inspire (I)}

\section{Current state-of-the-art planning algorithms}

\section{What after? Any new methods (A)}

\section{Present goal of paper (I\&A)}

\section{Is it time to reconsider classical planners under uncertainty (I\&A)}

\bibliographystyle{plain}
\bibliography{Plan_presentation_monday}

\end{document}
